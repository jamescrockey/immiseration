
\subsection{Hours worked \label{sec:hours}}
One possibility is that stagnant earnings reflect in part reductions in hours worked. This alters the comparison across generations since we normally presume that welfare is decreasing in hours worked. Figure~\ref{fig:hours_combine} reports the number of hours usually worked per week over the life course for each generation. Looking at the plots for men in the left column we see that, consistent with existing evidence \citep{Blundell2011,McDaniel2011}, that there have been no abrupt changes in the number of hours worked. There is some evidence that Silent Generation high school graduates worked more on average and particularly in their 30s, and that Millennials seem to work less than Boomers and Gen. X'ers, but the overall differences are relatively small. There are, as expected, greater changes for women. With a clear increase in hours worked by all generations subsequent to the Silent Generation for all women. As well as smaller, but still noticeable, differences for college-educated women between the Boomers and Gen. X'ers (and Millennials).  Figure~\ref{fig:hours_race} in the Appendix provides analogous plots for African Americans and Hispanic Americans showing similar patterns.  Overall, it seems that there has not been a sufficiently large decrease in hours worked by American men to imply a rising real hourly wage. 


%We return to this in section~\ref{sec:hedonic}


\begin{figure}[ht]
\caption{Weekly hours worked by generation over the life cycle}\label{fig:hours_combine}
\includegraphics[width=\linewidth]{hoursgenlc_combine.pdf}
\subcaption*{\emph{Source:} ASEC supplement of the Current Population Survey (CPS), survey years 1976-2018 \\
\emph{Notes:} Includes the total population and individual weights are used. `College' includes those who attended college and have at least a bachelor's degree. The vertical axis is median real wage in \$1000, measured in 1999 dollars. The measure of hours worked is usual hours worked per week. }
\end{figure}

The righthand panels of Figure~\ref{fig:hours_combine} show that the average hours worked by women was lower for the Silents than it was for subsequent generations. However, this change was minimal for high school graduates, and while larger only $1-3$ hours per week for those who attended college. Thus suggesting, that any increases in median female earnings are driven by increased hourly wages rather than increases in hours worked.
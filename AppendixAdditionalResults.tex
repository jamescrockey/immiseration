
\begin{figure}[h]
    \caption{Percentiles of the earnings distribution by year}\label{fig:percearnings} 
    \includegraphics[width=13cm]{percentilesofwage.pdf}
    \subcaption*{\emph{Source:} ASEC supplement of the Current Population Survey (CPS), survey years 1962-2018 \\
    \emph{Notes:} Includes the total population, wages are adjusted for inflation, and individual weights are used.  The vertical axis is median real income in \$1000, measured in 1999 dollars.}
    \end{figure}


%\begin{figure}[h]
%\caption{Maximum Median Wage (in \$1000) by Year born and the Age which it was reached.}\label{fig:maxmedwagecombine} 
%\includegraphics[width=13cm]{maxmedwage_combine.eps}
%\subcaption*{\emph{Source:} Current Population Survey (CPS)}
%\end{figure}
\subsection{Age of Peak Earnings}
The blue line in the top-left panel of Figure~\ref{fig:maxmedinc_combine54} plots the maximum median wage reached by year born for male high school graduates. We then annotate these points (in green) with the age when this income was reached.  We see that median American born in 1945 had maximum earnings of just over $\$65,000$ which they achieved age 47. In comparison, the maximum median wage of those born 10--15 years later was substantially lower, but was achieved by their early 30s. Those born from around 1961 to 1970 not only had lower maximum earnings but they did not receive them until they were 50. More recent cohorts had again lower maximum earnings, albeit slightly earlier at ages 42--44. Given the effects of the Financial Crisis, it maybe premature to reach a conclusion about those born in the late 1970s as it is conceivable that their earnings will still increase meaningfully. 

\begin{figure}[ht]
\caption{Maximum Median Wage (in \$1000) by Year Born}\label{fig:maxmedinc_combine54} 
\includegraphics[width=\linewidth]{maxmedinc_combine.pdf}
\subcaption*{\emph{Source:} ASEC supplement of the Current Population Survey (CPS), survey years 1962-2018 \\
\emph{Notes:} Includes the total population, wages are adjusted for inflation, and individual weights are used. The vertical axis is median real wage in \$1000, measured in 1999 dollars. The line is the maximum median wage by each year born and the label refers to the age when this maximum median wage was achieved. 'College' includes those who attended college and have at least a bachelor's degree.}
\end{figure}

\subsection{By Year}

\begin{figure}[h]
\caption{Median Income (in \$1000) for each Generation over time}\label{fig:medinccombine} 
\includegraphics[width=13cm]{incgen_combine.pdf}
\subcaption*{\emph{Source:} ASEC supplement of the Current Population Survey (CPS), survey years 1962-2018 \\
\emph{Notes:} Includes the total population, wages are adjusted for inflation, and individual weights are used. `College' includes those who attended college but didn't graduate, those did graduate, and those who have an advanced degree. The vertical axis is median real income in \$1000, measured in 1999 dollars.}
\end{figure}



\subsection{Hours worked \label{sec:apphours}}
One possibility is that stagnant earnings reflect in part reductions in hours worked. This alters the comparison across generations since we normally presume that welfare is decreasing in hours worked. Figure~\ref{fig:hours_combine} reports the number of hours usually worked per week over the life course for each generation. Looking at the plots for men in the left column we see that, consistent with existing evidence \citep{Blundell2011,McDaniel2011}, that there have been no abrupt changes in the number of hours worked. There is some evidence that Silent Generation high school graduates worked more on average and particularly in their 30s, and that Millennials seem to work less than Boomers and Gen. X'ers, but the overall differences are relatively small. There are, as expected, greater changes for women. With a clear increase in hours worked by all generations subsequent to the Silent Generation for all women. As well as smaller, but still noticeable, differences for college-educated women between the Boomers and Gen. X'ers (and Millennials).  Figure~\ref{fig:hours_race} provides analogous plots for African Americans and Hispanic Americans showing similar patterns.  Overall, it seems that there has not been a sufficiently large decrease in hours worked by American men to imply a rising real hourly wage. 


%We return to this in section~\ref{sec:hedonic}


\begin{figure}[ht]
\caption{Weekly hours worked by generation over the life cycle}\label{fig:hours_combine}
\includegraphics[width=\linewidth]{hoursgenlc_combine.pdf}
\subcaption*{\emph{Source:} ASEC supplement of the Current Population Survey (CPS), survey years 1976-2018 \\
\emph{Notes:} Includes the total population and individual weights are used. `College' includes those who attended college and have at least a bachelor's degree. The vertical axis is median real wage in \$1000, measured in 1999 dollars. The measure of hours worked is usual hours worked per week. }
\end{figure}

The righthand panels of Figure~\ref{fig:hours_combine} show that the average hours worked by women was lower for the Silents than it was for subsequent generations. However, this change was minimal for high school graduates, and while larger only $1-3$ hours per week for those who attended college. Thus suggesting, that any increases in median female earnings are driven by increased hourly wages rather than increases in hours worked.
\begin{figure}[h]
\caption{Hours worked by generation over the life cycle}\label{fig:hours_race}
	\includegraphics[width=13cm]{hourslcrace_combine.pdf}
\subcaption*{\emph{Source:} ASEC supplement of the Current Population Survey (CPS), survey years 1962-2018 \\
\emph{Notes:} Includes the total population, wages are adjusted for inflation, and individual weights are used. The vertical axis is usual hours worker per week.}
\end{figure}


\begin{figure}
\centering
\caption{Share of Generation in each Occupation Group at Age 30\label{fig:AppOcc}}
\begin{subfigure}{0.49\textwidth}
\caption{Managerial \& Professional Specialty}
\includegraphics[width=\linewidth]{share_byocc90ly_occgr1.pdf}
\end{subfigure}%
\begin{subfigure}{0.49\linewidth}
\caption{Technical Sales \& Administrative Support} 
\includegraphics[width=\linewidth]{share_byocc90ly_occgr2.pdf}
\end{subfigure}\\
\begin{subfigure}{0.49\linewidth}
\caption{Service}
\includegraphics[width=\linewidth]{share_byocc90ly_occgr3.pdf}
\end{subfigure}%
\begin{subfigure}{0.49\textwidth}
\caption{Farming, Forestry \& Fishing} 
\includegraphics[width=\linewidth]{share_byocc90ly_occgr4.pdf}
\end{subfigure}
\begin{subfigure}{0.49\textwidth}
\caption{Precision Production, Craft \& Repair}
\includegraphics[width=\linewidth]{share_byocc90ly_occgr5.pdf}
\end{subfigure}
\begin{subfigure}{0.49\textwidth}{}
\caption{Operaters, Fabricators \& Labourers }
\includegraphics[width=\linewidth]{share_byocc90ly_occgr6.pdf}
\end{subfigure}
\subcaption*{\emph{Source:} ASEC supplement of the Current Population Survey (CPS) \\}
\end{figure}

In this paper we document how, when comparing one generation to another, the median real wages of American men and women have been declining since the Silent Generation, born between 1925--1945. This is in contrast to consistent output and productivity growth over the same period \citep{Jorgenson2008,Gordon2017}. This phenomenon of \emph{declining} incomes, first documented by \cite{Guvenen2017}, is shown to be true conditional on a broad set of controls and allowing for unrestricted heterogeneity across industries. With the exception of the lowest-earning women and non-Whites it is true across the income distribution. It has two key sets of implications. Firstly, this lack of intergenerational progress may, as argued by \cite{freidman:2005}, lead to an increasingly challenging environment for democracy. Secondly, given consistent productivity growth it implies that the labour share of income has been falling on a generational basis. We investigate this possibility and show that, within sector and state, and conditional on a range of controls, that the labour share is lower for Gen. X'ers and Millennials.  

Of course, to some degree this is implied by the findings of \cite{Karabarbounis2008,Piketty2014,Autor2018a,autoretal:2020} that the labour share is decreasing, but what is novel is that the labour share is systematically different for workers of different generations, even conditioning on age, year and occupation. What it is that has caused this change is not something this paper speaks to, but we note that the change is consistent across most industries and for all sub-groups. 

Furthermore, it is not immediately obvious which of the leading explanations such as the rise of `superstar firms' \citep{autoretal:2020}, automation \cite{Autor2018a}, the price of investment goods \citep{Karabarbounis2008}, or the rate of productivity growth \citep{Grossman2018}, or the rise of offshoring \citep{Elsby2013}, would predict such large changes between generations, other things equal. 
Many of us take economic progress for granted. We expect that normally our earnings will increase from one year to the next and that we will be more prosperous than our parents and grandparents were. Yet, this expectation is increasingly misaligned with recent experience. US real median earnings have seen little improvement since 1980.\footnote{See Figure~\ref{fig:percearnings} in the Appendix.} Meanwhile, US GDP per capita has nearly doubled since 1980.\footnote{Measured in 2010 Dollars, it was $\$28,589$ in 1980 and $\$54,551$ in 2018. Data from: \url{https://data.worldbank.org/indicator/NY.GDP.PCAP.KD?locations=US}} This increase reflects both growth in women's market earnings due to greater labour market participation as well as increases in the earnings and other income of the affluent.  

This paper studies this stagnation from an intergenerational perspective. Given conventional lifecycle income trajectories such stagnation in median earnings implies declines in total lifecycle earnings for younger generations. We trace the real earnings of each generation over the life cycle and document that for each generation subsequent to the Baby Boomers, living standards have declined substantially in real terms. That is, rather than being richer than their parents, the median member of Generation X, born between 1965--1979, or a Millennial born in the period 1980--1999, is poorer at every point during their working lives than their parents were as members of either the Boomer or Silent Generations, born between 1946--1964 or 1925--1945 respectively. 

This can be seen in Figure~\ref{fig:medwage_tot_age} which shows median earnings in $1999$ dollars at each age for white, male, high school educated, Americans by decade of birth. Comparing the median wage across cohorts, we see that those born in the 1920s, were earning over $\$30,000$ in by their early 30s. This is less than those born in the 1930s and the 1940s, but interestingly this cohort have the highest peak earnings of any cohort, at around $\$55,000$.  However, the 1940s cohort saw their wages drop by nearly a quarter in real terms at the beginning of the 1980s relative to their wages in the preceding decade and never recover. A similar change seems to affect the previous cohorts, but later in life where it is conflated with retirement. Cohorts from the 1950s onwards, see comparatively little wage growth, earning less at every point in their lives than their forebears. For example, a white male high school graduate  born in the 1930s is earning about $\$50,000$ by age 40, their son, born in the 1950s, makes $\$40,000$, their grandson born in the 1970s had a median wage around $\$35,000$. \cite{chetty:2014} documents the growing importance of the `birth lottery', our results imply the average ticket is now a losing one in absolute terms. 

Yet such an aggregate decline may mask important forms of heterogeneity. Our data cover the period from the early 1960s to 2018, during which discrimination on the basis of sex, race, and subsequently disability were made illegal.
The period also saw considerable growth in women's labour market participation. 
It also saw considerable increases in educational outcomes, alongside changes in the sectoral composition of employment. We show however, that despite all of these advances that the phenomenon of declining wages is not limited to white men\footnote{X study... } , or to those without college degrees.\footnote{See for example...} 
Indeed the scale of the intergenerational-decline was sufficient to offset any gains from reductions in inequality. 
Both high school and college educated women experienced declines in real earnings in every generation subsequent to the Silent generation, as did Black and Hispanic men. 
While, the average incomes of Millenial and Gen. X Black and Hispanic women exceed those of their Silent or Millenial equivalents, these appear to be driven by improvements in educational outcomes. 
The wages of a Black female, college or high school, graduate are no higher for those born in 1981 than 1945. 
Moving beyond simple demographic groups quantile regression estimates show that we observe these declines across the wage distribution even when conditioning on a broad range of observable characteristics. Specifically, we find that Boomers' earnings were slightly lower than the Silent Generation's, whilst Gen. X'ers and Millennials were earning $6\%$ and $12\%$ less than the Silent Generation, respectively.

Such declines in the wages of most Americans imply one or both of two changes. 
Either, that most American workers have become less productive with each generation subsequent to the Silent generation. 
Perhaps, because of changes in the sectoral composition of the workforce.\footnote{For example.} 
Or, alternatively, that the decline in the labor share has been sufficient to more than offset productivity growth.\footnote{Recent papers documenting the decline in the US labor share include. \textbf{HERE}} 
To investigate this we construct generation-specific labor-share time-series which we in turn disaggregate further by industry, sex, and in the final section, union-status. 
These data reveal both that the aggregate decline in the labour share masks important heterogeneity. While some industries such as manufacturing, as has been previously documented, have seen large declines in the labor share, other such as finance and the utilities sector have seen substantial increases. The data also suggest that the decline in women's labor share has been at least as large as that of men, and from a higher starting point. One explanation for this is that women's labor share was initially high despite discrimination due to lower rates of labour force participation. \textbf{ARE WE SURE ABOUT THE DATA AND THIS ARGUMENT...}

We close the paper by asking why the labour share fell so markedly. In particular, we focus on differences in unionisation across generations and exposure to import \textbf{DO WE WANT TO GET INTO THIS?}. The results suggest that...

This paper is organised as follows.\textbf{The next section...}
% The first is that, since the silent generation, that labour productivity has been 

% %As such it is related to and builds on the recent work of \cite{Guvenen2017}, discussed in detail below, who are the first to our knowledge to document this phenomenon. 
%  The first section of this paper discusses intergenerational changes in earnings, providing evidence of similar patterns, with few exceptions, across demographic groups, and the population as a whole. In particular, we find that the median earnings of male college graduates have also declined. This is also true for women with the exception of the Silent Generation, as well as African American and Hispanic men. The key exception has been the substantial improvement in the earnings of African American and Hispanic women. Both graphical analysis at the cohort level, as well individual level regression estimates, show that this is a general phenomenon. Considering conditional demographic and educational controls, we find that Boomers' earnings were slightly lower than the Silent Generation's, whilst Gen. X'ers and Millennials were earning $6\%$ and $12\%$ less than the Silent Generation, respectively.%\footnote{We also show that, as well as earnings being lower, later generations have had to wait longer to attain peak earnings.}
%Figure 1

\begin{figure}[ht]
  \caption{Median wage of white male high school graduates, by decade of birth, over the life cycle}\label{fig:medwage_tot_age} 
  \includegraphics[width=\linewidth]{lifecyclebycohort_whitemenHS.pdf}
  \subcaption*{\emph{Source:} ASEC supplement of the Current Population Survey (CPS), survey years 1962-2018 \\
  \emph{Notes:} Includes the male population between the ages of 23 and 65 who are high school graduates and have wages above the defined minimum income threshold. Wages are adjusted for inflation and individual weights are used.}
  \end{figure}
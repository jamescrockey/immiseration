
\begin{figure}[ht]
    \caption{Median wage (in \$1000) by generation over the life cycle}\label{fig:medwage_tot} 
    \includegraphics[width=\linewidth]{wagegenlc_combine.pdf}
    \subcaption*{\emph{Source:} ASEC supplement of the Current Population Survey (CPS), survey years 1962-2018 \\
    \emph{Notes:} Includes the total population, wages are adjusted for inflation, and individual weights are used. `College' includes those who attended college and have at least a bachelor's degree. The vertical axis is median real wage in \$1000, measured in 1999 dollars.}
    \end{figure}
  
  %Figure 3 
  
  \begin{figure}[h]
    \caption{Median wage (in \$1000) for each generation over time}\label{fig:medwagecombine} 
    \includegraphics[width=\linewidth]{wagegen_combine.pdf}
    \subcaption*{\emph{Source:} ASEC supplement of the Current Population Survey (CPS), survey years 1962-2018 \\
    \emph{Notes:} Includes the total population, wages are adjusted for inflation, and individual weights are used. `College' includes those who attended college and have at least a bachelor's degree. The vertical axis is median real wage in \$1000, measured in 1999 dollars.}
    \end{figure}
  
  
  
  
  %Figure 5
  
  \begin{figure}[ht]
    \caption{Median wage by generation over the life cycle}\label{fig:medwagelc_race}
    \includegraphics[width=\linewidth]{racelc_combine.pdf}
    \subcaption*{\emph{Source:} ASEC supplement of the Current Population Survey (CPS), survey years 1962-2018 \\
    \emph{Notes:} Includes the total population, wages are adjusted for inflation, and individual weights are used. The vertical axis is median real wage in \$1000, measured in 1999 dollars.}
    \end{figure}
  
  
  (Hours, Age max salary-> Appendix)


  We will use the March supplement of the Current Population Survey (CPS) for the bulk of our analysis, as well as the Economic Census. Our main variable of interest here is annual wage and salary income.\footnote{We use wage and earnings interchangeably to refer to total annual income from wages and salaries.} Further discussion of these data and how we handle them may be found in Appendix~\ref{sec:data_appendix}. 


\subsection{Wages}


\subsubsection{Lifetime wages}

%\rb{In this section we report results which extend those of \cite{Guvenen2017}, making use of the detailed demographic data in the CPS to disaggregate by race and education as well as gender. We also work with a slightly different sample. \cite{Guvenen2017} mostly focus on those cohorts who were 25 in 1957 to those who were 25 in 1983, such that they can follow each cohort for the ages 25--55, and define lifetime earnings over that period. This has the important advantage that education is largely complete by age 25, and fewer people retire or otherwise leave the workforce before age 55. We choose to focus on a longer time span -- on those born between 1925 and 1994 -- at the price of not being able to follow the last generation, the Millennials, throughout their lives. However, by now the oldest of these are nearly 40 and thus we are in position to compare their comparative fortunes to this point. This comparison is worthwhile because by around this age many Americans would hope to have bought a house, started a family, and to be earning close to their maximum real incomes. More generally, a standard discounting argument implies that an income profile that offers greater earnings early in one's life, holding total earnings constant, is to be preferred. Thus, the shape and level of each generation's earnings profile matters for welfare, and may be usefully compared by age 40. }

In this section we present results which disaggregate on a number of characteristics. Making use of the detailed demographic data in the ASEC supplement of the March CPS, we are able to present results which disaggregate by race and education as well as gender. We focus on those born between 1925 and 1994 -- at the price of not being able to follow the last generation, the Millennials, throughout their lives. However, by now the oldest of these are nearly 40 and thus we are in position to compare their comparative fortunes to this point. This comparison is worthwhile because by around this age many Americans would hope to have bought a house, started a family, and to be earning close to their maximum real incomes. More generally, a standard discounting argument implies that an income profile that offers greater earnings early in one's life, holding total earnings constant, is to be preferred. Thus, the shape and level of each generation's earnings profile matters for welfare, and may be usefully compared by age 40.


\begin{table}[h!]
\begin{threeparttable}
\caption{Different birth cohorts.\label{cohortdef}}
\begin{tabular}{l|l}   
\hline
\textbf{Year born} & \textbf{Birth cohort} \\
\hline
2000 -- Present & Generation Z 			\\
1980 -- 1999 	& Millennials (Gen. Y) \\
1965 -- 1979 	& Generation X (Gen. X)\\
1946 -- 1964  	& Baby Boomers (Boomers)\\
1925 --	1945	& Silent Generation 	\\	
\hline
\end{tabular}
\end{threeparttable}
\end{table}


For exposition purposes, we divide the Americans in our data into generations, as they are typically defined. Table~\ref{cohortdef} displays how we define each generation. The top-left panel of Figure~\ref{fig:medwage_tot} is analogous to Figure~\ref{fig:medwage_tot_age} except now the median wage over the life cycle is plotted by generation.\footnote{In this section, we do not comment on the statistical significance of differences observed. For evidence that this differences are indeed significant, refer to Section \ref{sec:regressions}, which includes a number of regressions.} We can see clearly that the Silent Generation (born 1925--1945 denoted by blue circles) have higher earnings at every point than the Boomers (purple diamonds), Gen. X'ers (green triangles) and Millennials (brown squares). Moreover, this difference is substantial, nearly $\$20,000$ a year at age 45, or two thirds of Boomer earnings. While the Boomers earn less than the Silent Generation, they do earn more than the later two generations. Moreover, they hit peak earnings sooner, by their late 20s, while Gen. X'ers experienced a much slower growth in their earnings, even if they seem to have converged by age 50. This is also true for Millennials. Figure~\ref{fig:medwagecombine} reports the same data but now with year rather than age on the horizontal axis.\footnote{Note, that there will be some difference in the estimates since Figure~\ref{fig:medwage_tot} takes the median of all members of a given generation at a given age. Figure~\ref{fig:medwagecombine} reports the median in a given year of all members of a generation who will hence be of a range of ages.} This makes clear the declining fortunes of high school graduates. Each generation's  median wage at each age is below (excluding a drop off in earnings for the Silent Generation from age 50 onwards) that of the one before. The average across all generations, not plotted, thus declines as Gen. X'ers and Millennials start to replace the Silent Generation and Boomers. Note, that we might expect, given substantial economic growth, the opposite: that each generation would start from a higher point than the preceding one and increase from there such that the curves would intersect. 

 The bottom-left panel of Figure~\ref{fig:medwage_tot} presents results for men with at least some college education. We see that, again, the median wage of the Silent Generation is higher at all points in their career. This means that the decline of real wages has not only been experienced by high school graduates. Suggesting that the phenomenon is not limited to those in lower-skill occupations.  But the difference with the Boomers is smaller now and there is no appreciable difference between the Boomers and the subsequent generations. This is consistent with skills-biased technological change advantaging those with more formal education in subsequent generations relative to those with less in their generation and thus reducing the gap between generations of the more educated. This explanation is also consistent with what seems to be some evidence of improved earnings for Millennials who attended college in the last couple of years relative to the Boomers and Gen. X'ers. But, without more data, it is not possible to rule out that this is just a short-term fluctuation.  

\begin{figure}[ht]
\caption{Median wage (in \$1000) by generation over the life cycle}\label{fig:medwage_tot} 
\includegraphics[width=\linewidth]{wagegenlc_combine.pdf}
\subcaption*{\emph{Source:} ASEC supplement of the Current Population Survey (CPS), survey years 1962-2018 \\
\emph{Notes:} Includes the total population, wages are adjusted for inflation, and individual weights are used. `College' includes those who attended college and have at least a bachelor's degree. The vertical axis is median real wage in \$1000, measured in 1999 dollars.}
\end{figure}

Looking at the bottom-left panel of Figure~\ref{fig:medwagecombine} again reinforces the point. We again can see lower earnings at every point for each subsequent generation and more notably for this sample, pronounced generational differences in the rate of progress over the life cycle. This can be seen by comparing the difference between the Boomer's curve and that of Gen. X or the Millennials, which are substantially flatter at the beginning. 

\begin{figure}[h]
\caption{Median wage (in \$1000) for each generation over time}\label{fig:medwagecombine} 
\includegraphics[width=\linewidth]{wagegen_combine.pdf}
\subcaption*{\emph{Source:} ASEC supplement of the Current Population Survey (CPS), survey years 1962-2018 \\
\emph{Notes:} Includes the total population, wages are adjusted for inflation, and individual weights are used. `College' includes those who attended college and have at least a bachelor's degree. The vertical axis is median real wage in \$1000, measured in 1999 dollars.}
\end{figure}

%% change the legend here ! 


\subsubsection{By gender}

The right two panels of Figure~\ref{fig:medwage_tot} show the results of the same analysis for women. Looking first at the results for high school graduates in the top-right panel, it is clear that women's median earnings are on average, across the life cycle and across all generations, considerably lower than those of men. It is also clear that there is little progress across generations. This result is in contrast to the findings of \cite{Guvenen2017} and this may reflect differences in the origins of the data used and the sample definition.\footnote{One feature of \cite{Guvenen2017} is that they are able to use administrative data providing recorded rather than self-reported earnings data. A disadvantage of this is that it may exclude unrecorded earnings, which our data should capture.} Looking at the top right panel of Figure~\ref{fig:medwagecombine} we see that each generation seems to converge to within a few thousand dollars of a median of $\$20,000$. 

The bottom-right panel of Figure~\ref{fig:medwage_tot} now shows the results for women who attended college. Here, the opposite story is true. Each generation seems to be out earning the one before it. Thus, the Silent Generation now has the lowest median wage, followed by the Boomers, the Gen. X'ers and finally Millennials. Consistent with this, in Figure~\ref{fig:medwagecombine} we now see this pattern of the median earnings of each cohort intersecting with those before it (albeit not yet for Millennials).  This suggests, that perhaps the growth in women's earnings documented by \cite{Guvenen2017}, are due solely to the growth in the earnings of college-educated women and the growth in the proportion of women attending college.\footnote{\cite{Guvenen2017} restrict the sample to those with consistent labour market engagement and a minimal level of income that may disproportionately exclude less-educated women, who may be more likely to be in informal employment.}	


\subsubsection{By race}

Another key margin of income inequality is race: African Americans and Hispanic Americans continue to have lower average incomes than other Americans \citep{Fryer2011}. Given that around the beginning of our sample period, the passage of the Civil Rights Act made discrimination on the basis of race illegal, and recent evidence suggests that discrimination can account for a relatively small proportion of the racial earnings gap \citep{Fryer2011}. Thus, we might expect subsequent generations of African American and Hispanic men to have higher incomes than those of the Silent Generation even if male earnings in general are declining. Similarly, we expect more rapid growth in the earnings of African American and Hispanic women. However, inspection of the top two panels of Figure~\ref{fig:medwagelc_race} which reproduces Figure~\ref{fig:medwage_tot} for African American and Hispanic men suggests that this is not the case. Incomes at each point in the life cycle are broadly constant across all four generations of African American men. It is unclear why relative pay of the Hispanic Silent Generation was so much higher than subsequent generations, but focusing on the Boomers onwards we see no evidence of an increase in the wages of Hispanic men either, and indeed arguably a decline. Of course, migration makes comparisons across generations more difficult and it maybe that the lack of earnings growth is due to a composition effect. This would explain, potentially, the substantial decline in earnings from the Silent Generation to subsequent generations. %\textbf{Is there a way to investigate this - say by looking at citizenship?} 

The bottom two panels report results for African American and Hispanic women, respectively. Now, we see clear signs of increasing incomes from one generation to the next. Looking first at the evidence for the African American women in the bottom left panel we see that working women of the Silent Generation earned around $\$5,000$ less than Boomers, who in turn earned less, albeit not as much less than Gen. X'ers and Millennials. A similar, but arguably more pronounced pattern can be seen in the bottom right panel for Hispanic women. Now, as well as daylight between the Silent Generation and the Boomers there is a clear difference between Boomers and Gen. X'ers and in turn them and Millennials. Common to both African American and Hispanic women is that Gen. X'ers, and particularly Millennials, both show signs of rapid income growth during their 20s and 30s. This is consistent with the closing of the gap in college enrolment rates in both populations compared to American women as a whole.  


\begin{figure}[ht]
\caption{Median wage by generation over the life cycle}\label{fig:medwagelc_race}
\includegraphics[width=\linewidth]{racelc_combine.pdf}
\subcaption*{\emph{Source:} ASEC supplement of the Current Population Survey (CPS), survey years 1962-2018 \\
\emph{Notes:} Includes the total population, wages are adjusted for inflation, and individual weights are used. The vertical axis is median real wage in \$1000, measured in 1999 dollars.}
\end{figure}